\selectlanguage{english}
%--------------------------------------------------------------------------------------
% Short formal and content guidelines
%--------------------------------------------------------------------------------------

\footnotesize
\begin{center}
\large
\textbf{\Large General information, the structure of the diploma plan}\\
\end{center}

The structure of the diploma plan at the BME Faculty of Electrical Engineering and Informatics:
\begin{enumerate}
\item Hardcover (with “BSc Thesis”, Author’s name, and year)
\item Thesis task description
\item Title page
\item Table of content
\item Student’s declaration about individual work and usage of electronic data 
\item Summary of the thesis work in English and in a second language
\item Introduction: explanation of the task, design objectives, motivations behind the task, short description of the organization of the thesis 
\item Detailed explanation and analysis of the task description
\item Preliminaries (results available in the literature, similar designs and constructions), comparisons and conclusions
\item Detailed description of the design process, evaluation of available options, motivations and justifications of design decisions
\item Critical assessment of the engineering product designed, further development options  
\item Acknowledgments (if applicable)
\item Detailed list of references
\item Annex(es)

\end{enumerate}

The contents of the \LaTeX diploma draft template document starting on the next page can be used.

The diploma plan should be printed on standard A4 sheets. The pages should be made with mirror margins (2.5~cm everywhere, 1~cm binding on the left side). The default font is 12-point Times New Roman, double-spaced, but this can be slightly deviated from, or other fonts are also allowed.

Every page -- except for the first four structural elements -- must have the page number.

Chapters must be arranged in decimal numbers. The figures must be inserted in the appropriate place, each chapter must be given a decimal number and an expressive title. The chapters are numbered with decimal subdivisions, a maximum of 3 subdivisions deep (e.g. 2.3.4.1.). Figures, tables and formulas should be numbered separately for each chapter (e.g. Figure 2.4, Table 4.2 or formula (3.2)). Align the chapter titles to the left, and use line alignment for the normal text. Center the figures, tables and their title. The title should be located under the marked part.

The images should preferably be created with a drawing program, and the equations should be written using an equation editor (\LaTeX~ provides obvious solutions for this).

The bibliography can be referenced in-text by serial numbering (this is the preferred solution) or in the Harvard system (by specifying the author and the year). The complete list should be listed at the end of the text in chronological order (in the case of numbered literary references, in reference order). However, the titles of the literary sources must always be given in the original language, possibly with the translation in parentheses. All publications in the list must be cited in the text (the \LaTeX template handles all this automatically with the help of Bib\TeX~). All publications include the following data after the authors: for journal articles, the exact title, title of the journal, year, issue, page number from to. The titles of journals should only be shortened if they are very well-known or very long. When entering Internet links, it is important to specify the owner and content of the page before the access path (since the link may even become inaccessible after a while), as well as the time of access.

\vspace{5mm}
Important:
\begin{itemize}
	\item The statement of the thesis creator/diploma designer (with the text content included in this template) is a mandatory requirement, without this, the thesis/diploma design cannot be criticized or defended at our school!
	\item Both the thesis and the attachment can be a maximum of 15~MB in size!
\end{itemize}

\vspace{5mm}
\begin{center}
Our faculty wishes you to enjoy the time spent with your thesis work!
\end{center}

\normalsize
\selectthesislanguage
